% Created 2023-03-25 Sat 11:42
% Intended LaTeX compiler: pdflatex
\documentclass[11pt,a4paper,final]{article}
\usepackage[a4paper, total={7in, 10in}]{geometry}
\usepackage{algorithm2e}
\usepackage{booktabs}
\usepackage{hyperref}
\usepackage{subcaption}
\usepackage{graphicx}
\usepackage{tikz}
\usepackage[utf8]{inputenc}
\usepackage[T1]{fontenc}
\usepackage{graphicx}
\usepackage{longtable}
\usepackage{wrapfig}
\usepackage{rotating}
\usepackage[normalem]{ulem}
\usepackage{amsmath}
\usepackage{amssymb}
\usepackage{capt-of}
\usepackage{hyperref}
\theoremstyle{definition}
\newtheorem{definition}{Definition}[section]
\author{Alexander Brown}
\date{\today}
\title{}
\hypersetup{
 pdfauthor={Alexander Brown},
 pdftitle={},
 pdfkeywords={},
 pdfsubject={},
 pdfcreator={Emacs 28.2 (Org mode 9.5.5)},
 pdflang={English}}
\begin{document}

\tableofcontents


\section{Preliminaries}
\label{sec:org24728ce}

\subsection{Mixed Interger Linear Program}
\label{sec:orgf576321}
A mixed integer linear programming (MILP) problem is a class of constrained optimization in which one seeks to find a
set of continuous or integer values that maximizes or minimizes an objective function while satisfying a set of
constraints \cite{chen-2010-applied}. Given an objective function \(J\), decision variables (i.e. variables of
optimization) \(x_j \in \mathbb{R}\) and \(y_k = \in \mathbb{Z}^+\), and input parameters \(c_j, d_k, a_{ij}, g_{ik}, b_i \in \mathbb{R}\), a MILP has the
mathematical structure represented in \autoref{eq:milp-structure} \cite{chen-2010-applied}.

\begin{equation}
\label{eq:milp-structure}
\begin{array}{lll}
\text{Maximize}   & J = \sum_j c_j x_j + \sum_k d_k y_k            &                 \\
\text{subject to} & \sum_j a_{ij} x_j + \sum_k g_{ik} y_k  \le b_i & (i = 1,2,...,m) \\
                  & x_j \ge 0                                      & (j = 1,2,...,n) \\
                  & y_k \in \mathbb{Z^+}0                          & (k = 1,2,...,n) \\
\end{array}
\end{equation}

This formulation of the MILP is also referred to as crisp. By this it is meant that each variable in the formulation
acts as an injective mapping to its number representation. In other words, no values on the formulation are fuzzy
\cite{kaur-2016-introd-fuzzy}.

\subsection{Fuzzy Sets and LR Fuzzy Numbers}
\label{sec:org93888dc}
This section introduces the notion of fuzzy numbers and some basic definitions. Let's begin with what a fuzzy number is
not. A classical (crisp) set is defined as a collection of elements \(x \in X\). Each element either belongs in the set, or
it does not \cite{zimmermann-2001-fuzzy-set}. For a fuzzy set, what is known as the characteristic functions applies
various degrees of membership for elements of a given set \cite{zimmermann-2001-fuzzy-set}.

\begin{definition}
Let $X$ be a collection of objects denoted generically by $x$, then a fuzzy set $\tilde{A}$ in $X$ is a set of ordered
pairs

\begin{equation}
\tilde{A} = \{(x, \mu_{\tilde{A}}(x))| x\in X\}
\end{equation}

\noindent
$\mu_{\tilde{A}}$ is called the membership function where $\mu_{\tilde{A}}$ is the mapping $\mu_{\tilde{A}} : X
\rightarrow [0,1]$; which assigns a real number to the interval $[0,1]$. The value of $\mu_{\tilde{A}}$ represents the
degree of membership of $x$ in $\tilde{A}$.
\end{definition}

\bibliographystyle{acm}
\bibliography{../../literature/ref,../../library/ref}
\end{document}
