% Created 2023-07-18 Tue 19:54
% Intended LaTeX compiler: pdflatex
\documentclass[11pt,a4paper,final]{article}
\usepackage[a4paper, total={7in, 10in}]{geometry}
\usepackage{algorithm2e}
\usepackage{booktabs}
\usepackage{subcaption}
\usepackage{graphicx}
\usepackage{tikz}
\usepackage[utf8]{inputenc}
\usepackage[T1]{fontenc}
\usepackage{graphicx}
\usepackage{longtable}
\usepackage{wrapfig}
\usepackage{rotating}
\usepackage[normalem]{ulem}
\usepackage{amsmath}
\usepackage{amssymb}
\usepackage{capt-of}
\usepackage{hyperref}
\newtheorem{definition}{Definition}[section]
\author{Alexander Brown}
\date{\today}
\title{Bus Charging Schedule Simulated Annealing with MILP Constraints}
\hypersetup{
 pdfauthor={Alexander Brown},
 pdftitle={Bus Charging Schedule Simulated Annealing with MILP Constraints},
 pdfkeywords={},
 pdfsubject={},
 pdfcreator={Emacs 28.2 (Org mode 9.6.7)}, 
 pdflang={English}}
\begin{document}

\maketitle
\tableofcontents


\section{Preliminaries}
\label{sec:org67d644d}
\subsection{Mixed Integer Linear Program}
\label{sec:org1e7ac49}
A mixed integer linear programming (MILP) problem is a class of constrained optimization in which one seeks to find a
set of continuous or integer values that maximizes or minimizes an objective function while satisfying a set of
constraints (Chen, Der-San and Batson, Robert G and Dang, Yu, 2010). Given an objective function \(J\), decision variables (i.e. variables of
optimization) \(x_j \in \mathbb{R}\) and \(y_k \in \mathbb{Z}^+\), and input parameters \(c_j, d_k, a_{ij}, g_{ik}, b_i \in \mathbb{R}\), a MILP has the
mathematical structure represented in autoref:eq:milp-structure (Chen, Der-San and Batson, Robert G and Dang, Yu, 2010).

\begin{equation}
\label{eq:milp-structure}
\begin{array}{lll}
\text{Maximize}   & J = \sum_j c_j x_j + \sum_k d_k y_k            &                 \\
\text{subject to} & \sum_j a_{ij} x_j + \sum_k g_{ik} y_k  \le b_i & (i = 1,2,...,m) \\
                  & x_j \ge 0                                      & (j = 1,2,...,n) \\
                  & y_k \in \mathbb{Z^+}0                          & (k = 1,2,...,n) \\
\end{array}
\end{equation}

This formulation of the MILP is also referred to as ``crisp''. By this it is meant that each variable in the formulation
acts as an injective mapping to its number representation. In other words, no values on the formulation are fuzzy
(Jagdeep Kaur and Amit Kumar, 2016).

\subsection{Fuzzy Sets Theory}
\label{sec:org71bd545}
This section introduces the notion of fuzzy numbers and some basic definitions. Concepts from this section are pulled
from (H.-J. Zimmermann, 2001, Sapan Kumar Das and T. Mandal and S. A. Edalatpanah, 2016, M. Yaghobi and M. Rabbani and M. Adabitabar Firozja and J. Vahidi, 2014, Marilyn Bello and Gonzalo N{\'a}poles and Ivett Fuentes and Isel Grau and Rafael Falcon and Rafael Bello and Koen Vanhoof, 2019).

\subsubsection{Fuzzy Sets}
\label{sec:org99f2be1}
Let's begin with what a fuzzy number is not. A classical (crisp) set is defined as a collection of elements \(x \in X\).
Crisp sets are binary, either an element either belongs in the set, or it does not (H.-J. Zimmermann, 2001).
For a fuzzy set, what is known as the characteristic functions applies various degrees of membership for elements of a
given set (H.-J. Zimmermann, 2001).

\begin{definition}
Let \(X\) be a collection of objects (often called the universe of discourse (Marilyn Bello and Gonzalo N{\'a}poles and Ivett Fuentes and Isel Grau and Rafael Falcon and Rafael Bello and Koen Vanhoof, 2019)). If \(X\) is denoted
generically by \(x\), then a fuzzy set \(\tilde{A}\) in \(X\) is a set of ordered pairs as shown in autoref:eq:membership-function.

\begin{equation}
\label{eq:membership-function}
\tilde{A} = \{(x, \mu_{\tilde{A}}(x))| x\in X\}
\end{equation}

\noindent
\(\mu_{\tilde{A}}\) is called the membership function where \(\mu_{\tilde{A}}\) is the mapping \(\mu_{\tilde{A}} : X \rightarrow
[0,1]\); which assigns a real number to the interval \([0,1]\). The value of \(\mu_{\tilde{A}}\) represents the degree of
membership of \(x\) in \(\tilde{A}\).
\end{definition}

This paper will use fuzzy sets defined on the real numbers \(\mathbb{R}\). The membership function describes the shape of
the fuzzy number. As an example, consider the following definition.

\begin{definition}
A fuzzy number that is represented by \(\tilde{A} = (a,b,c)\) is said to be triangular if its membership function is
defined as autoref:eq:triangular-fuzzy-number.

\begin{equation}
\label{eq:triangular-fuzzy-number}
  \mu_{\tilde{A}}(x) =
  \begin{cases}
    \frac{(x-a)}{(b-a)} & a \le x \le b \\
    \frac{(d-x)}{(d-b)} & c \le x \le d \\
    0                   & \text{otherwise}
  \end{cases}
\end{equation}
\end{definition}

\begin{definition}
The fuzzy set \(tilde{A}\) in \(\mathbb{R}\) is normal if \(\text{max}_x \mu_{\tilde{A}}(x) = 1\).
\end{definition}

\begin{definition}
A fuzzy set \(\tilde{A}\) in \(\mathbb{R}\) is convex if and only if the membership function of \(\tilde{A}\) satisfies the inequality

\begin{equation*}
\mu_{\tilde{A}}[\beta x_1 + (1-\beta)x_2] \ge \text{min}[\mu_{\tilde{A}}(x_1), \mu_{\tilde{A}}(x_2)]\; \forall x_1, x_2 \in \mathbb{R}\; \beta \in [0,1]
\end{equation*}
\end{definition}

\begin{definition}
A fuzzy number is a normal convex fuzzy set in \(\mathbb{R}\).
\end{definition}

\begin{definition}
The triangular fuzzy number \(\tilde{A}\) is nonnegative \(\iff\; a \ge 0\).
\end{definition}

A more general definition of fuzzy numbers is known as LR fuzzy numbers
(Jagdeep Kaur and Amit Kumar, 2016, H.-J. Zimmermann, 2001).

\begin{definition}
A function \(L:[0,\infty] \rightarrow [0,1]\) (or \(R:[0,\infty] \rightarrow [0,1]\)) is said to be reference a function of the fuzzy number if and only
if

\begin{enumerate}
\item \(L(0) = 1\) (or \(R(0) = 1\))
\item \(L\) (or \(R\)) is non-increasing on \([0,\infty)\)
\end{enumerate}
\end{definition}

\begin{definition}
A fuzzy number \(\tilde{A}\) defined on the set of real numbers, \(\mathbb{R}\), denoted as \((m,n,\alpha,\beta)_{LR}\), is said to be an \(LR\)
flat fuzzy number if its membership function \(\mu_{\tilde{A}}(x)\) is defined as

\begin{equation}
\mu_{\tilde{A}}(x) =
\begin{cases}
L(\frac{m-x}{\alpha}) & x \le m, \alpha > 0 \\
R(\frac{m-n}{\beta}) & x \ge m, \beta > 0 \\
1                & m \le x \le n
\end{cases}
\end{equation}
\end{definition}

\begin{definition}
An \(LR\) flat fuzzy number \(\tilde{A} = (m,n,\alpha,\beta)_{LR}\) is said to be a non-negative \(LR\) flat fuzzy number if and only
if \(m-\alpha \ge 0\) and is said to be non-positive \(LR\) flat fuzzy number if and only if \(m - \alpha \le 0\) is a real number.
\end{definition}

\begin{definition}
An \(LR\) flat fuzzy number \(\tilde{A} = (m,n,\alpha,\beta)_{LR}\) is said to be an unrestricted \(LR\) flat fuzzy number if and only
if \(m - \alpha\) is a real number.
\end{definition}

\subsubsection{Fuzzy Arithmetic}
\label{sec:orgcfb0958}
If two triangular fuzzy numbers \(\tilde{a}_1 = \{a_1, a_2, a_3\}\) and \(\tilde{b}_1 = \{b_1, b_2, b_3\}\) are nonnegative
then the following operations are defined in autoref:eq:fuzzy-arithmetic.

\begin{equation}
\label{eq:fuzzy-arithmetic}
\begin{array}{lcl}
\tilde{a} \oplus \tilde{b} & = & (a_1 + b_1, a_2 + b_2, a_3 + b_3) \\
\tilde{a} \ominus \tilde{b} & = & (a_1 + b_3, a_2 + b_2, a_3 + b_1) \\
\tilde{a} \otimes \tilde{b} & = & (a_1 b_1, a_2 b_2, a_3 b_3)       \\
\end{array}
\end{equation}

\subsubsection{Comparing Fuzzy Numbers}
\label{sec:org425f88c}
Fuzzy numbers do not directly provide a method of ordering nor do they always provide an ordered set like real numbers
(Marilyn Bello and Gonzalo N{\'a}poles and Ivett Fuentes and Isel Grau and Rafael Falcon and Rafael Bello and Koen Vanhoof, 2019). There are multiple methods for ordering fuzzy numbers, each coming with advantages and
disadvantages. Different properties have been applied to justify comparison of fuzzy numbers, such as: preference,
rationality, and robustness (Mariano Jim{\'e}nez and Mar Arenas and Amelia Bilbao and M. Victoria Rodrı´guez, 2007, Marilyn Bello and Gonzalo N{\'a}poles and Ivett Fuentes and Isel Grau and Rafael Falcon and Rafael Bello and Koen Vanhoof, 2019, Jagdeep Kaur and Amit Kumar, 2016). These
methods are commonly known as ranking functions or ordering functions
(Marilyn Bello and Gonzalo N{\'a}poles and Ivett Fuentes and Isel Grau and Rafael Falcon and Rafael Bello and Koen Vanhoof, 2019, Sapan Kumar Das and T. Mandal and S. A. Edalatpanah, 2016, Jagdeep Kaur and Amit Kumar, 2016). Commonly, including in this work, the First
index of Yager (Ronald R. Yager, 1981) is used (autoref:eq:first-index-yager).

\begin{equation}
\label{eq:first-index-yager}
\mathfrak{R}(\tilde{A}) = \frac{\sum_i a_i}{|\tilde{A}|}
\end{equation}

\noindent
where \(|\cdot|\) represents the cardinality of the fuzzy number. In words, autoref:eq:first-index-yager is merely the average
of the values in the fuzzy number.

\subsection{Fully Fuzzy Linear Programming}
\label{sec:orga7b71a3}
Much like the MILP, fully fuzzy linear programs (FFLP) it is a class of constrained optimization in which one seeks to
find a set of continuous variables that either maximizes or minimizes an objective function, \(J\), while satisfying a set
of constraints. The key difference in FFLP is that it is designed to accommodate imprecise information
(Marilyn Bello and Gonzalo N{\'a}poles and Ivett Fuentes and Isel Grau and Rafael Falcon and Rafael Bello and Koen Vanhoof, 2019, Jagdeep Kaur and Amit Kumar, 2016). In FFLP, the parameters and decision variables are fuzzy and
linear. A general FFLP is represented as shown in autoref:eq:general-fflp. The subscripts \(\cdot_e\), \(\cdot_l\), and \(\cdot_g\)
indicate to equality, less than, and greater than constraints, respectively.

\begin{equation}
\label{eq:general-fflp}
\begin{array}{lll}
\text{Maximize}   & J = \sum_j \tilde{C}_j \otimes \tilde{X}_j              &                 \\
\text{subject to} & \sum_j \tilde{a}_{ej} \otimes \tilde{x}_j = \tilde{b}_e &  \forall e = 1,2,3,... \\
                  & \sum_j \tilde{a}_{lj} \otimes \tilde{x}_j \le \tilde{b}_l &  \forall l = 1,2,3,... \\
                  & \sum_j \tilde{a}_{gj} \otimes \tilde{x}_j \ge \tilde{b}_l &  \forall g = 1,2,3,...
\end{array}
\end{equation}

There are many methods of solving FFLP
(Marilyn Bello and Gonzalo N{\'a}poles and Ivett Fuentes and Isel Grau and Rafael Falcon and Rafael Bello and Koen Vanhoof, 2019, Jagdeep Kaur and Amit Kumar, 2016, Ali Ebrahimnejad, 2016, Nasseri, SH and Behmanesh, E and Taleshian, F and Abdolalipoor, M and TAGHI, NEZHAD NA, 2013); however, most
solution methods convert the fuzzy model into a crisp model that can be solved using traditional methods
(Marilyn Bello and Gonzalo N{\'a}poles and Ivett Fuentes and Isel Grau and Rafael Falcon and Rafael Bello and Koen Vanhoof, 2019). In (Nasseri, SH and Behmanesh, E and Taleshian, F and Abdolalipoor, M and TAGHI, NEZHAD NA, 2013, Marilyn Bello and Gonzalo N{\'a}poles and Ivett Fuentes and Isel Grau and Rafael Falcon and Rafael Bello and Koen Vanhoof, 2019), the method of converting the FFLP
into a crisp MILP is simply by applying the ranking function to the objective function and breaking the constraints down
into a set of crisp constraints as shown in autoref:eq:nasseri-solution. The constraints are separated according to the
definition of fuzzy set multiplication defined in autoref:eq:fuzzy-arithmetic. The fuzzy number index is represented is
the exponent rather than the subscript to clearly distinguish between the indexed value in the fuzzy number and the
constraint index (i.e. \(\tilde{A} = (a^1,a^2,a^3)\)). Furthermore, it is assumed that the fuzzy numbers are nonnegative.

\begin{equation}
\label{eq:nasseri-solution}
\begin{array}{lll}
\text{Maximize}   & J = \mathfrak{R}\Big(\sum_j (c_j^1,c_j^2,c_j^3)(x_j^1,x_j^2,x_j^3)\Big) &\\
\text{subject to} & \sum_j a_{ej}^1 x_j^1 = b_e^1 &  \forall e = 1,2,3,... \\
                  & \sum_j a_{lj}^1 x_j^1 \le b_l^1 &  \forall l = 1,2,3,... \\
                  & \sum_j a_{gj}^1 x_j^1 \ge b_l^1  &  \forall g = 1,2,3,... \\
                  & \sum_j a_{ej}^2 x_j^2 = b_e^2 &  \forall e = 1,2,3,... \\
                  & \sum_j a_{lj}^2 x_j^2 \le b_l^2 &  \forall l = 1,2,3,... \\
                  & \sum_j a_{gj}^2 x_j^2 \ge b_l^2  &  \forall g = 1,2,3,... \\
                  & \sum_j a_{ej}^3 x_j^3 = b_e^3 &  \forall e = 1,2,3,... \\
                  & \sum_j a_{lj}^3 x_j^3 \le b_l^3 &  \forall l = 1,2,3,... \\
                  & \sum_j a_{gj}^3 x_j^3 \ge b_l^3  &  \forall g = 1,2,3,... \\
                  & x_j^2 - x_j^1 \ge 0         & x_j^3 - x_j^2 \ge 0 \\
\end{array}
\end{equation}

\noindent
To be more succinct, the FFLP can also equivalently be written as autoref:eq:nasseri-solution-condensed.

\begin{equation}
\label{eq:nasseri-solution-condensed}
\begin{array}{lll}
\text{Maximize}   & J = \mathfrak{R}\Big(\sum_j (c_j^1,c_j^2,c_j^3) \otimes (x_j^1,x_j^2,x_j^3)\Big) &\\
\text{subject to} & \sum_j a_{ej}^k x_j^k = b_e^k &  \forall e = 1,2,3,... \\
                  & \sum_j a_{lj}^k x_j^k \le b_l^k &  \forall l = 1,2,3,... \\
                  & \sum_j a_{gj}^k x_j^k \ge b_l^k  &  \forall g = 1,2,3,... \\
                  & x_j^2 - x_j^1 \ge 0         & x_j^3 - x_j^2 \ge 0 \\
                  & \forall k \in \{1,2,...\}        &                  \\
\end{array}
\end{equation}

Where \(k\) has a max value equal to the cardinality to the type of fuzzy number being utilized. This can be further be
elaborated on by rewriting the inequality constraints as equality constraints by introducing slack as equality
constraints by introducing slack as equality constraints by introducing slack variables. This is useful as it
represents the formulation in a standard form (Chen, Der-San and Batson, Robert G and Dang, Yu, 2010, Robert J. Vanderbei, 2020). It also has the
slightly less useful benefit of (mostly) providing the solver a set of equations called a hyperplane (Chen, Der-San and Batson, Robert G and Dang, Yu, 2010).

The given method is called the Kumar and Kaurs method. Generally speaking, it is designed to solve FFLP problems with
inequality constraints having LR flat fuzzy numbers. Given the FFLP autoref:eq:general-fflp and assuming that
\(\tilde{x}_j\) is an LR flat fuzzy number, the problem can be reformulated as autoref:eq:kumar-kaurs-fuzzy
(Jagdeep Kaur and Amit Kumar, 2016).

\begin{equation}
\label{eq:kumar-kaurs-fuzzy}
\begin{array}{lll}
\text{Maximize}   & J = \sum_j \tilde{C}_j \otimes \tilde{X}_j              &                                              \\
\text{subject to} & \sum_j \tilde{a}_{ej} \otimes \tilde{x}_j               = \tilde{b}_e & \forall e = 1,2,3,...                \\
                  & \sum_j \tilde{a}_{lj} \otimes \tilde{x}_j \oplus \tilde{S}_l = \tilde{b}_l \oplus \tilde{S'}_l & \forall l = 1,2,3,... \\
                  & \sum_j \tilde{a}_{gj} \otimes \tilde{x}_j \oplus \tilde{S}_e = \tilde{b}_l \oplus \tilde{S'}_g & \forall g = 1,2,3,... \\
                  & \mathfrak{R}(\tilde{S_l}) - \mathfrak{R}(\tilde{S_l'}) \ge 0                                     & \forall l = 1,2,3,...      \\
                  & \mathfrak{R}(\tilde{S_g}) - \mathfrak{R}(\tilde{S_g'}) \le 0                                     & \forall g = 1,2,3,...
\end{array}
\end{equation}

Expanding the set of equation and using the condensed notation in autoref:eq:nasseri-solution-condensed we find
autoref:eq:kumar-kaurs-crisp (Jagdeep Kaur and Amit Kumar, 2016).

\begin{equation}
\label{eq:kumar-kaurs-crisp}
\begin{array}{lll}
\text{Maximize}    & J = \mathfrak{R}\Big(\sum_j (c_j^1,c_j^2,c_j^3) \otimes (x_j^1,x_j^2,x_j^3)\Big) &                       \\
\text{subject to}  & \sum_j a_{ej}^k x_j^k = b_e^k                                   &  \forall e = 1,2,3,...      \\
                   & \sum_j a_{lj}^k x_j^k s_l^k \le s_l^{'k} b_l^k                    &  \forall l = 1,2,3,...      \\
                   & \sum_j a_{gj}^k x_j^k s_g^k \ge s_l^{'k} b_l^k                    &  \forall g = 1,2,3,...      \\
                   & \mathfrak{R}(\tilde{S_l}) - \mathfrak{R}(\tilde{S_l'}) = 0                         & \forall l = 1,2,3,...       \\
                   & \mathfrak{R}(\tilde{S_g}) - \mathfrak{R}(\tilde{S_g'}) = 0                         & \forall g = 1,2,3,...       \\
                   & x_j^2 - x_j^1 \ge 0                                            & x_j^3 - x_j^2 \ge 0     \\
                   & s_j^2 - s_j^1 \ge 0                                            & s_j^3 - s_j^2 \ge 0     \\
                   & s_j^{'2} - s_j^{'1} \ge 0                                      & s_j^{'3} - s_j^{'2} \ge 0 \\
                   & \forall k \in \{1,2,...\}                                            &                       \\
\end{array}
\end{equation}

\section{The Crisp BAP and PAP}
\label{sec:org8564cc9}
\subsection{The Berth Allocation Problem}
\label{sec:orgcf54e50}
The BAP models the optimal distribution of container ships to terminals in order to be serviced. The allocation of the
ships depends primarily on the size of the ship and its service time
(Pablo Frojan and Juan Francisco Correcher and Ramon Alvarez-Valdes and Gerasimos Koulouris and Jose Manuel Tamarit, 2015, Akio Imai and Etsuko Nishimura and Stratos Papadimitriou, 2001, Katja Buhrkal and Sara Zuglian and Stefan Ropke and Jesper Larsen and Richard Lusby, 2011). Most BAP models assume the service
time, size, and preferred terminals to be the input parameters and have delay, deviation from ideal position to be the decision
variables (Pablo Frojan and Juan Francisco Correcher and Ramon Alvarez-Valdes and Gerasimos Koulouris and Jose Manuel Tamarit, 2015, Akio Imai and Etsuko Nishimura and Stratos Papadimitriou, 2001, Katja Buhrkal and Sara Zuglian and Stefan Ropke and Jesper Larsen and Richard Lusby, 2011). A general formulation for the
BAP of a single quay is described in autoref:eq:generalbap. The variables are as described in autoref:tab:bapvariables.

The equations will now explained. autoref:subeq:bapobj is the objective function for the BAP. In this form, it is
attempting to minimize the total time from arrival to service completion. autoref:subeq:baptemporal is a big-M
constraint that is used to check if ship \(i\)'s service time ends before ship \(i\). That is \(\sigma_{ij}=1\) if \(a_j \ge a_i -
s_i\) and \(\sigma_{ij} = 0\) otherwise. Similarly, autoref:subeq:bapspatial checks if ship \(i\) is asbelow ship \(j\). That is
\(\psi_{ij} = 1\) if \(v_j \ge v_i - s_i\) and \(\psi_{ij} = 0\) otherwise. The equations autoref:subeq:bapvalidpos -
autoref:subeq:bappsi ensure that ship \(j\) is either assigned after ship \(i\) has finished its service and/or \(j\) is
assigned below ship \(i\); however, \(\sigma_{ij} = \sigma_{ji} \ne 1\) or \(\psi_{ij} = \psi_{ji} \ne 1\). That is to say a ship cannot be queued
before and after another or be queued above and below another simultaneously. autoref:subeq:bapdetach represents the time the ship will depart
from the berth after being serviced. autoref:subeq:bapvalidtime ensures that the arrival time is less than the starting ervice time and that the
starting service time is early enough such that the total time spent servicing the vessel is less than the time horizon. autoref:subeq:bapspaces
defines the sets for each decision variable.

\begin{subequations}
\label{eq:generalbap}
\label{eq:bapconstrs}
\begin{align}
    \text{Minimize }   & \sum_{i=1}^I (e_i - a_i)                                       \label{subeq:bapobj}    \\
    \text{subject to } &a_j - a_i - s_i - (\sigma_{ij} - 1)T \geq 0                         \label{subeq:baptemporal}         \\
                       &v_j - v_i - s_i - (\psi_{ij} - 1)S \geq 0                         \label{subeq:bapspatial}        \\
                       &\sigma_{ij} + \sigma_{ji} + \psi_{ij} + \psi_{ji} \geq 1                       \label{subeq:bapvalidpos}    \\
                       &\sigma_{ij} + \sigma_{ji} \leq 1                                         \label{subeq:bapsigma}        \\
                       &\psi_{ij} + \psi_{ji} \leq 1                                         \label{subeq:bappsi}        \\
                       &s_i + a_i = e_i                                             \label{subeq:bapdetach}       \\
                       &a_i \leq u_i \leq (T - s_i)                                       \label{subeq:bapvalidtime} \\
                       &\sigma_{ij} \in \{0,1\},\;\psi_{ij} \in \{0,1\}\; v_i \in [0 \mbox{ } S ] \label{subeq:bapspaces}
\end{align}
\end{subequations}

\begin{table}[htbp]
\caption{\label{tab:bapvariables}Table of variables used for the BAP}
\centering
\begin{tabular}{ll}
\textbf{Variable} & \textbf{Description}\\[0pt]
\hline
Input constants & \\[0pt]
\(I\) & Number of total ships\\[0pt]
\hline
Input variables & \\[0pt]
\(a_i\) & Arrival time of ship \(i\)\\[0pt]
\(e_i\) & Time ship \(i\) must departs the berth\\[0pt]
\hline
Decision Variables & \\[0pt]
\(\psi_{ij}\) & Tracks spatial overlap for ships \((i,j)\)\\[0pt]
\(\sigma_{ij}\) & Tracks temporal overlap for ships \((i,j)\)\\[0pt]
\(s_i\) & Service time for ship \(i\)\\[0pt]
\(u_i\) & Service start time for ship \(i\)\\[0pt]
\(v_i\) & Assigned quay for ship \(i\)\\[0pt]
\hline
\end{tabular}
\end{table}

\subsection{The Position Allocation Problem (from MILP paper)}
\label{sec:org84ad2cc}
The BAP formulation forms the basis of the PAP; however, there are some differences in the way the variables are
perceived. Using the same formulation as autoref:eq:generalbap, the \(i^{th}\) visit, the starting service time, 
\(u_i\), is now the starting charge time, the berth location, \(v_i\), is now the charger queue for assignment, 
and the service time, \(s_i\), is now the time to charge. The PAP utilizes a number of parameters. The following 
parameters are constants.

\begin{itemize}
\item \(Q\)   : charger length
\item \(T\)   : time horizon
\item \(N\)   : number of incoming vehicles
\item \(s_i\) : charging time for vehicle \(i;\; 1 \leq i \leq N\)
\item \(a_i\) : arrival time of vehicle \(i;\; 1 \leq i \leq N\)
\end{itemize}

These constants define the problem bounds. The following list provides a series of decision variables used in the
formulation.

\begin{itemize}
\item \(u_i\)         : starting time of service for vehicle \(i;\; 1 \leq i \leq N\)
\item \(v_i\)         : charge location \(i;\; 1 \leq i \leq N\)
\item \(e_i\)         : departure time for vehicle \(i;\; 1 \leq i \leq N\)
\item \(\sigma_{ij}\) : binary variable that determines ordering of vehicles \(i\) and \(j\) in time
\item \(\psi_{ij}\)   : binary variable that determines relative position of vehicles \(i\) and \(j\) when charging simultaneously
\end{itemize}

\section{References}
\label{sec:orga7cbbd7}
\noindent
Akio Imai and Etsuko Nishimura and Stratos Papadimitriou (2001). \emph{The Dynamic Berth Allocation Problem for a Container Port}, Elsevier {BV}.

\noindent
Ali Ebrahimnejad (2016). \emph{New Method for Solving Fuzzy Transportation Problems With Lr Flat Fuzzy Numbers}, Information Sciences.

\noindent
Chen, Der-San and Batson, Robert G and Dang, Yu (2010). \emph{Applied integer programming}, Wiley.

\noindent
H.-J. Zimmermann (2001). \emph{Fuzzy Set Theory-and Its Applications}, Springer Netherlands.

\noindent
Jagdeep Kaur and Amit Kumar (2016). \emph{An Introduction to Fuzzy Linear Programming Problems}, Springer International Publishing.

\noindent
Katja Buhrkal and Sara Zuglian and Stefan Ropke and Jesper Larsen and Richard Lusby (2011). \emph{Models for the Discrete Berth Allocation Problem: a Computational Comparison}, Transportation Research Part E: Logistics and Transportation Review.

\noindent
M. Yaghobi and M. Rabbani and M. Adabitabar Firozja and J. Vahidi (2014). \emph{Comparison of Fuzzy Numbers With Ranking Fuzzy and Real Number}, Journal of Mathematics and Computer Science.

\noindent
Mariano Jim{\'e}nez and Mar Arenas and Amelia Bilbao and M. Victoria Rodrı´guez (2007). \emph{Linear Programming With Fuzzy Parameters: an Interactive Method Resolution}, European Journal of Operational Research.

\noindent
Marilyn Bello and Gonzalo N{\'a}poles and Ivett Fuentes and Isel Grau and Rafael Falcon and Rafael Bello and Koen Vanhoof (2019). \emph{Fuzzy Activation of Rough Cognitive Ensembles Using OWA Operators}, Springer International Publishing.

\noindent
Nasseri, SH and Behmanesh, E and Taleshian, F and Abdolalipoor, M and TAGHI, NEZHAD NA (2013). \emph{Fully fuzzy linear programming with inequality constraints}, INTERNATIONAL JOURNAL OF INDUSTRIAL MATHEMATICS.

\noindent
Pablo Frojan and Juan Francisco Correcher and Ramon Alvarez-Valdes and Gerasimos Koulouris and Jose Manuel Tamarit (2015). \emph{The Continuous Berth Allocation Problem in a Container Terminal With Multiple Quays}, Expert Systems with Applications.

\noindent
Robert J. Vanderbei (2020). \emph{Linear Programming}, Springer International Publishing.

\noindent
Ronald R. Yager (1981). \emph{A Procedure for Ordering Fuzzy Subsets of the Unit Interval}, Information Sciences.

\noindent
Sapan Kumar Das and T. Mandal and S. A. Edalatpanah (2016). \emph{A Mathematical Model for Solving Fully Fuzzy Linear Programming Problem With Trapezoidal Fuzzy Numbers}, Applied Intelligence.
\end{document}