% Created 2023-03-26 Sun 09:42
% Intended LaTeX compiler: pdflatex
\documentclass[11pt,a4paper,final]{article}
\usepackage[a4paper, total={7in, 10in}]{geometry}
\usepackage{algorithm2e}
\usepackage{booktabs}
\usepackage{hyperref}
\usepackage{subcaption}
\usepackage{graphicx}
\usepackage{tikz}
\usepackage[utf8]{inputenc}
\usepackage[T1]{fontenc}
\usepackage{graphicx}
\usepackage{longtable}
\usepackage{wrapfig}
\usepackage{rotating}
\usepackage[normalem]{ulem}
\usepackage{amsmath}
\usepackage{amssymb}
\usepackage{capt-of}
\usepackage{hyperref}
\newtheorem{definition}{Definition}[section]
\author{Alexander Brown}
\date{\today}
\title{Bus Charging Schedule Simulated Annealing with MILP Constraints}
\hypersetup{
 pdfauthor={Alexander Brown},
 pdftitle={Bus Charging Schedule Simulated Annealing with MILP Constraints},
 pdfkeywords={},
 pdfsubject={},
 pdfcreator={Emacs 28.2 (Org mode 9.5.5)},
 pdflang={English}}
\begin{document}

\maketitle
\tableofcontents


\section{Preliminaries}
\label{sec:org0360fb3}

\subsection{Mixed Interger Linear Program}
\label{sec:org89328a9}
A mixed integer linear programming (MILP) problem is a class of constrained optimization in which one seeks to find a
set of continuous or integer values that maximizes or minimizes an objective function while satisfying a set of
constraints \cite{chen-2010-applied}. Given an objective function \(J\), decision variables (i.e. variables of
optimization) \(x_j \in \mathbb{R}\) and \(y_k \in \mathbb{Z}^+\), and input parameters \(c_j, d_k, a_{ij}, g_{ik}, b_i \in \mathbb{R}\), a MILP has the
mathematical structure represented in \autoref{eq:milp-structure} \cite{chen-2010-applied}.

\begin{equation}
\label{eq:milp-structure}
\begin{array}{lll}
\text{Maximize}   & J = \sum_j c_j x_j + \sum_k d_k y_k            &                 \\
\text{subject to} & \sum_j a_{ij} x_j + \sum_k g_{ik} y_k  \le b_i & (i = 1,2,...,m) \\
                  & x_j \ge 0                                      & (j = 1,2,...,n) \\
                  & y_k \in \mathbb{Z^+}0                          & (k = 1,2,...,n) \\
\end{array}
\end{equation}

This formulation of the MILP is also referred to as ``crisp''. By this it is meant that each variable in the formulation
acts as an injective mapping to its number representation. In other words, no values on the formulation are fuzzy
\cite{kaur-2016-introd-fuzzy}.

\subsection{Fuzzy Sets Theory}
\label{sec:orgaa03d88}
This section introduces the notion of fuzzy numbers and some basic definitions. Concepts from this section are pulled
from \cite{zimmermann-2001-fuzzy-set,das-2016-mathem-model,yaghobi-2014-compar-fuzzy,bello-2019-fuzzy-activ}.

\subsubsection{Fuzzy Sets}
\label{sec:org2d2f9a6}
Let's begin with what a fuzzy number is
not. A classical (crisp) set is defined as a collection of elements \(x \in X\). Crisp sets are binary, either an element
either belongs in the set, or it does not \cite{zimmermann-2001-fuzzy-set}. For a fuzzy set, what is known as the
characteristic functions applies various degrees of membership for elements of a given set.

\cite{zimmermann-2001-fuzzy-set}.
\begin{definition}
Let \(X\) be a collection of objects (often called the universe of discourse \cite{bello-2019-fuzzy-activ}). If \(X\) is denoted
generically by \(x\), then a fuzzy set \(\tilde{A}\) in \(X\) is a set of ordered pairs as shown in  \autoref{eq:membership-function}.

\begin{equation}
\label{eq:membership-function}
\tilde{A} = \{(x, \mu_{\tilde{A}}(x))| x\in X\}
\end{equation}

\noindent
\(\mu_{\tilde{A}}\) is called the membership function where \(\mu_{\tilde{A}}\) is the mapping \(\mu_{\tilde{A}} : X \rightarrow
[0,1]\); which assigns a real number to the interval \([0,1]\). The value of \(\mu_{\tilde{A}}\) represents the degree of
membership of \(x\) in \(\tilde{A}\).
\end{definition}

This paper will use fuzzy sets defined on the real numbers \(\mathbb{R}\). The membership function describes the shape of
the fuzzy number. As an example, consider the following definition.

\begin{definition}
A fuzzy number that is represented by \(\tilde{A} = (a,b,c)\) is said to be triangular if its membership function is
defined as \autoref{eq:triangular-fuzzy-number}.

\begin{equation}
\label{eq:triangular-fuzzy-number}
  \mu_{\tilde{A}}(x) =
  \begin{cases}
    \frac{(x-a)}{(b-a)} & a \le x \le b \\
    \frac{(d-x)}{(d-b)} & c \le x \le d \\
    0                   & \text{else}
  \end{cases}
\end{equation}
\end{definition}

\begin{definition}
The fuzzy set \(tilte{A}\) in \(\mathbb{R}\) is normal if \(\text{max}_x \mu_{\tilde{A}}(x) = 1\).
\end{definition}

\begin{definition}
A fuzzy set \(\tilde{A}\) in \(\mathbb{R}\) is convex if and only if the membership function of \(\tilde{A}\) satisfies the inequality

\begin{equation*}
\mu_{\tilde{A}}[\beta x_1 + (1-\beta)x_2] \ge \text{min}[\mu_{\tilde{A}}(x_1), \mu_{\tilde{A}}(x_2)]\; \forall x_1, x_2 \in \mathbb{R}\; \beta \in [0,1]
\end{equation*}
\end{definition}

\begin{definition}
A fuzzy number is a normal convex fuzzy set in \(\mathbb{R}\).
\end{definition}

\begin{definition}
The triangular fuzzy number \(\tilde{A}\) is nonnegative \(\iff\; a \ge 0\).
\end{definition}

\subsubsection{Fuzzy Arithmetic}
\label{sec:orgc5a0c9f}
If two triangular fuzzy numbers \(\tilde{a}_1 = \{a_1, a_2, a_3\}\) and \(\tilde{b}_1 = \{b_1, b_2, b_3\}\) are nonnegative
then the following operations are defined

\begin{equation}
\begin{array}{lcl}
\tilde{a} \oplus \tilde{b} & = & (a_1 + b_1, a_2 + b_2, a_3 + b_3) \\
\tilde{a} \ominus \tilde{b} & = & (a_1 + b_3, a_2 + b_2, a_3 + b_1) \\
\end{array}
\end{equation}

\subsubsection{Comparing Fuzzy Numbers}
\label{sec:org5dc7d32}
Fuzzy numbers do not directly provide a method of ordering nor do they always provide an ordered set like real numbers
\cite{bello-2019-fuzzy-activ}. There are multiple methods for ordering fuzzy numbers, each coming with advantages and
disadvantages. Different properties have been applied to justify comparison of fuzzy numbers, such as: preference,
rationality, and robustness \cite{jimenez-2007-linear-progr,bello-2019-fuzzy-activ,kaur-2016-introd-fuzzy}. Commonly,
including in this work, the First index of Yager \cite{yager-1981-proced-order} is used (\autoref{eq:first-index-yager}).

\begin{equation}
\label{eq:first-index-yager}
\mathfrak{R}(\tilde{A}) = \frac{\sum_i a_i}{|\tilde{A}|}
\end{equation}

\noindent
where \(|\cdot|\) represents the cardinality of the fuzzy number. In words, \autoref{eq:first-index-yager} is merely the average
of the values in the fuzzy number.

\bibliographystyle{plain}
\bibliography{../../literature/ref,../../library/ref}

\subsection{Fully Fuzzy Linear Programming (FFLP)}
\label{sec:orgb7b0069}
\end{document}
